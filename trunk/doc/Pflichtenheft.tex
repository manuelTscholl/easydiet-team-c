%% LyX 1.6.9 created this file.  For more info, see http://www.lyx.org/.
%% Do not edit unless you really know what you are doing.
\documentclass[ngerman]{article}
\usepackage[T1]{fontenc}
\usepackage{CJKutf8}
\usepackage{array}

\makeatletter

%%%%%%%%%%%%%%%%%%%%%%%%%%%%%% LyX specific LaTeX commands.
%% Because html converters don't know tabularnewline
\providecommand{\tabularnewline}{\\}

\makeatother

\usepackage{babel}

\begin{document}
\begin{CJK}{UTF8}{}%
\flushright{EasyDiet} \flushright{Pflichtenheft (SRS - Software
Requirements Specification)} \flushright{Version \guilsinglleft{}1.0\guilsinglright{}}
\end{CJK}

\flushleft{} 


\section{Revisionsverlauf}

\begin{tabular}{|l|l|l|l|}
\hline 
Datum  & Version  & Beschreibung  & Author \tabularnewline
\hline 
03.03.2001  & 1.0  & Grundstruktur erstellen  & Manuel Tscholl \tabularnewline
\hline
\end{tabular}


\subsection{Subtitle}


\section{Einführung}

Manuel 


\subsection{System}


\subsection{Zweck}


\subsection{Umfang}


\subsection{Referenzen}


\subsection{Überblick}

Manuel ende

Stephan 


\section{Stakeholder- und Benuterbeschreibung}


\subsection{Überblick Stakeholder/Benutzer}

\begin{tabular}{|l|l|l|}
\hline 
Name & Rolle/Funktion & interessiert an \tabularnewline
\hline
\end{tabular}


\subsection{Benutzerumgebung}

Stephan ende


\section{Produkt Überblick}


\subsection{Zusammenfassung der Produktfähigkeiten/Eigenschaften}

\begin{flushleft}
Das Produkt EasyDiet soll die DiätassistentInnen bei der Erstellung
und Überwachung von Diäten unterstützen.
\par\end{flushleft}



\begin{tabular}{|>{\raggedright}p{8cm}|l|}
\hline 
\textbf{Produktfähigkeit/-eigenschaft} & \textbf{Stakeholder Nutzen/Gewinn}\tabularnewline
\hline 
Stammdatenverwaltung & \tabularnewline
\hline 
Erfassung von Befunden & \tabularnewline
\hline 
Erfassung des Patientenstatus & \tabularnewline
\hline 
Erfassung von persönlichen und gesundheitlichen Ausschließungskriterien
für Nahrungsmittel & \tabularnewline
\hline 
Erfassung von erhaltenen Parametern & \tabularnewline
\hline 
Anlegen neuer Parameter & \tabularnewline
\hline 
Führung eines Kontaktjournales für jeden Patienten & \tabularnewline
\hline 
Erfassung von Ernährungsprotokollen & \tabularnewline
\hline 
Auswertung von Ernährungsprotokollen & \tabularnewline
\hline 
Erstellung von Ernährungsempfehlungen & \tabularnewline
\hline 
Erstellung von Speiseplänen  & \tabularnewline
\hline 
Erstellung von Diätplänen & \tabularnewline
\hline 
Soll-Ist Analyse & \tabularnewline
\hline 
Berechnung von Über- bzw. Untergewicht & \tabularnewline
\hline 
Erstellung von Rezepten & \tabularnewline
\hline 
Auswertung von Rezepten & \tabularnewline
\hline 
Erstellung von Mahlzeiten & \tabularnewline
\hline 
Erstellung eines Rezeptbüchleins & \tabularnewline
\hline 
Erstellung einer Gut \& Schlecht Liste & \tabularnewline
\hline 
Drucken eines Erstellungsprotokollvordruckes & \tabularnewline
\hline 
Verwendung von Nährstofftabellen & \tabularnewline
\hline
\end{tabular}


\subsection{Produkt Fähigkeiten/Eigenschaften}


\subsubsection{Stammdatenverwaltung}


\subsubsection{Erfassung von Befunden}


\subsubsection{Erfassung des Patientenstatus}


\subsubsection{Erfassung von persönlichen und gesundheitlichen Auschließungskriterien
für Nahrungsmittel}


\subsubsection{Erfassungvon erhaltenen Parametern}


\subsubsection{Anlegen neuer Parameter}


\subsubsection{Führung eines Kontaktjournales für jeden Patienten}


\subsubsection{Erfassung von Ernährungsprotokollen}


\subsubsection{Auswertung von Ernährungsprotokollen}


\subsubsection{Erstellung von Ernährungsempfehlungen}


\subsubsection{Erstellung von Speiseplänen}


\subsubsection{Erstellung von Diätplänen}


\subsubsection{Soll-Ist Analyse}


\subsubsection{Berechnung von Über- bzw. Untergewicht}


\subsubsection{Erstellung von Rezepten}


\subsubsection{Auswertung von Rezepten}


\subsubsection{Erstellung von Mahlzeiten}


\subsubsection{Erstellung eines Rezeptbüchleins}


\subsubsection{Erstellung einer Gut \& Schlecht Liste}


\subsubsection{Drucken eines Erstellungsprotokollvordruckes}


\subsubsection{Verwendung von Nährstofftabellen}


\subsection{Annahmen und Abhängigkeiten}

Es muss eine Java Virtual Machine für das vom Kunden eingesetzte Betriebssystem
existieren. Die Verbindung zum Datenbankserver muss Vorhanden sein
und Adresse zu diesem Server muss ebenfalls bekannt sein. Das Betriebssystem
muss grafische Benutzeroberflächen unterstützen. Einer Drucker sollte
vorhanden sein.


\section{Domänenmodell}


\subsection{Überblick}


\subsection{Detailliertes Modell}


\subsubsection{eine Klasse}


\subsubsection{eine weitere Klasse}


\section{Dynamisches Modell}


\subsection{Detaillierte Benutzungsfälle (Usecases)}


\subsubsection{Detaillierte Benutzungsfall/Beschreibung}


\subsubsection{Sequenz Diagramm}


\subsubsection{Kontrakte}


\section{Nonfunktionale Anforderungen}


\subsection{Regeln}


\subsection{Usability}


\subsection{Zuverlässigkeit}


\subsection{Performanz}


\subsection{Unterstützbarkeit}


\subsection{Online Benutzerdokumentation und Help System}


\subsection{zugekaufte Komponenten}


\subsection{Schnittstellen}


\subsubsection{Benutzerschnittstellen}


\subsubsection{Software Schnittstellen}


\subsubsection{Kommunikationsschnittstellen}


\subsection{zusätzlicher Lizenzierungen}


\subsection{Copyright und andere rechtliche Anforderungen}


\subsection{Anzuwendende Standards}


\section{Iterationsplan (Timeboxes)}


\subsection{Überblick}


\subsection{1. Timebox}


\subsubsection{Benutzungsfälle (UseCases)}


\subsubsection{Architektur}


\subsubsection{Deliverables}


\subsubsection{Abhängikeiten}


\subsection{2. Timebox}


\subsubsection{Benutzungsfälle (UseCases)}


\subsubsection{Architektur}


\subsubsection{Deliverables}


\subsubsection{Abhängikeiten}


\subsection{3. Timebox}


\subsubsection{Benutzungsfälle (UseCases)}


\subsubsection{Architektur}


\subsubsection{Deliverables}


\subsubsection{Abhängikeiten}


\subsection{Glossar}
\end{document}
