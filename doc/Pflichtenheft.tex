\documentclass{article}

\usepackage [ngerman] {babel}
\usepackage [UTF8] {inputenc}

\begin{document}


\flushright{EasyDiet}
\flushright{Pflichtenheft  (SRS - Software Requirements Specification)}
\flushright{Version \flq1.0\frq}

\flushleft{}
\section{Revisionsverlauf}

\begin{tabular}{|l | l | l| l |}
\hline
Datum & Version & Beschreibung & Author \\ \hline
03.03.2001 & 1.0 & Grundstruktur erstellen & Manuel Tscholl \\ \hline
\end{tabular}
\subsection{Subtitle}

\section{Einführung}

Manuel
\subsection{System}
\subsection{Zweck}
\subsection{Umfang}
\subsection{Referenzen}
\subsection{Überblick}
Manuel ende

Stephan
\section{Stakeholder- und Benuterbeschreibung}
\subsection{\"Uberblick Stakeholder/Benutzer}
\begin{tabular}[]{l|l|l}
Name&Rolle/Funktion&interessiert an \\ \hline
\end{tabular}
\subsection{Benutzerumgebung}
Stephan ende

Michael, Ali, Fritz
\section{Produkt Überblick}
\subsection{Zusammenfassung der Produktfähigkeiten/Eigenschaften}
Das Produkt Easy-Diet soll die Diätassistentinen unterstützen in der Erstellung und Überwachung von Diäten.
\subsection{Produkt Fähigkeiten/Eigenschaften}
\subsubsection{Fähigkeit 1}
\subsubsection{Fähigkeit 2}
\subsection{Annahmen und Abhängigkeiten}
JVM muss auf dem Kunden OS lauffähig sein.
Verbindung zum Datenbankserver muss Vorhanden sein und Adresse muss ebenfalls bekannt sein.
Das OS muss GUIs unterstützen.
Drucker muss vorhanden.
M,A,F ende

\section{Domänenmodell}
\subsection{Überblick}
\subsection{Detailliertes Modell}
\subsubsection{eine Klasse}
\subsubsection{eine weitere Klasse}

\section{Dynamisches Modell}
\subsection{Detaillierte Benutzungsfälle (Usecases)}
\subsubsection{Detaillierte Benutzungsfall/Beschreibung}
\subsubsection{Sequenz Diagramm}
\subsubsection{Kontrakte}

\section{Nonfunktionale Anforderungen}
\subsection{Regeln}
\subsection{Usability}
\subsection{Zuverlässigkeit}
\subsection{Performanz}
\subsection{Unterstützbarkeit}
\subsection{Online Benutzerdokumentation und Help System}
\subsection{zugekaufte Komponenten}
\subsection{Schnittstellen}
\subsubsection{Benutzerschnittstellen}
\subsubsection{Software Schnittstellen}
\subsubsection{Kommunikationsschnittstellen}
\subsection{zusätzlicher Lizenzierungen}
\subsection{Copyright und andere rechtliche Anforderungen}
\subsection{Anzuwendende Standards}

\section{Iterationsplan (Timeboxes)}
\subsection{Überblick}
\subsection{1. timebox}
\subsubsection{Benutzungsfälle (UseCases)}
\subsubsection{Architektur}
\subsubsection{Deliverables}
\subsubsection{Abhängikeiten}

\subsection{2. Timebox}
\subsubsection{Benutzungsfälle (UseCases)}
\subsubsection{Architektur}
\subsubsection{Deliverables}
\subsubsection{Abhängikeiten}

\subsection{3. Timebox}
\subsubsection{Benutzungsfälle (UseCases)}
\subsubsection{Architektur}
\subsubsection{Deliverables}
\subsubsection{Abhängikeiten}

\subsection{Glossar}

\end{document}
